\chapter{Introduction}

\section{Background}
From the beginning of time finding information about particular areas or building was a kind of tricky job unless you knew where to go or whom to talk to about the history attached to these certain places. More recently we can access websites and see the information about a certain location or area. Usually, the Dublin county council office has the data on each town for example Lucan. With the suburbs growing steadily over the last number of years the changing of towns has been in an extraordinary state. Towns that have been around for hundreds of years are the main reason for the development of such an application (App) so that Dublin's history can be brought back to life.
\par
What the developer is willing to achieve is to allow people to navigate, explore and learn about the history of different buildings. Throughout the use of this application, the user's will then grasp a better knowledge of the area in which they live.
Throughout the development of the Android Operating System (OS) starting in 2006 steadily numbers grew after Google's release of sharing its operating system to all mobile manufacturer's. The first mobile to be released with such an OS was the HTC Dream.\cite{androidHtc}  which is why the developer has chosen to develop an android application.
With the introduction of the Google play market, which launched on the March 6th,  2012, There has been a huge demand for smart technology. Now we can use our mobile phone for doing just about anything, such as surfing the web, reading emails, even playing games and the interpretation of books. 


\section{Introduction of the InforMe@Dublin Project}

With regards to the following proposed project on which its aim is to provide additional information to tourists or just the common public who wish to learn more about the history of their local area and wish to view old photos or view historical places in their area. When the user is in this specific area pop-ups will appear that can be selected and information will then be display for that designated place. Users may post their findings of images whether that be past or present to allow others to learn from what users have posted.
\par
The information such as the history of the area will be sourced from a variety of on-line sources that will be compiled by the developer. The information will be accessed by the application using Firebase database which is a cloud storage service for Android, OS and web projects of which was created by Google.
\par
With the research that has been uncovered in the literature review of the following thesis. The developer has then chosen what technologies should be used in order to complete the InforMe@Dublin application.

\section{Research Question's}
The Research questions of which the developer must answer are defined below. Each of these questions will be answered throughout the following thesis.
\begin{enumerate}
	\item\textbf{\textit{Is there a need for an application like this on the marketplace?}}

	\item\textbf{\textit{Is there a way to save battery life while using GPS / Internet connection within an android application?}}

	\item\textbf{\textit{Will location services, and mobile internet take much battery or data?}} 

	\item\textbf{\textit{Will the APP provide a wiki type interface where visitors can add to the information about a historic site?}}
	
\end{enumerate}    





