\chapter{Testing and Evaluation}

\section{Description of the Chapter}
Regarding the following chapter, the developer will provide the testing of the application on which will prohibit errors running throughout at specific points and within the application. Some visual testing of the app has been conducted within which some guests were allowed to test upon the release of it to the Google play store. The smoothness of the application and latency issues, as well as any other problems which may have been discovered on completing such an app, would be delivered and results published.Testing any application which is due to be release needs vigorous testing some of the most tests related to application development are functional testing, usability testing, security testing as well as a not so complicated method of visually assessing the code of which was written. For an application to be release, a mobile application must be able to satisfy each of these testing methods.
\section{Functional Correctness}
\textbf{\textit{Does the functionality of the following project meet the standards of what the project is intended for?}}
\newline
The answer to the question above is yes, the project functions above the expectation of the developer which is why the developer has chosen to publish the application on the Google Play Store. All parts of the program perform well and do what the developer has intended the app in question to do. 
\par
Each action of which a user will choose will do what it is supposed to do and will not receive any unknown errors unless something drastic or problematic is wrong with a device on which intends to use or install the following app. The graphical display is easily understood but could use a how-to tutorial after the creation of an account within the app for the not so experienced users.
\par
With the exception of accepting permission's on the app as well as entering the user's details if these have not been entered an error on the text field would be shown graphically to the user.

\section{Loading of the Application}
Regarding the fast usability of the InforMe@Dublin application with regards to latency related issues, the following application has none of this kind. While the developer was coding the loading up of the application and authentication, the user just has to wait until the progress dialogue completes which means the user is in the process of being logged into the app which helps as a user is not trying to access certain UI elements while a long process is being completed. Overall the application has a smooth feel to it with no issues being related to latency issues. This may happen on some device with a low amount of ram available to the device.

\section{Loading of the Map}
While the map loads on the mobile device a camera animation is performed this allow's for a nice smooth feel to the user's location. Utilising these actions on the device leave's users with a great feel while accessing the app as it seems good quality overall with regards to scoping out the location of the device. Only errors will occur if the device cannot receive the geofences on which need to be received from the Firebase database.

\section{Loading of the Information}
Concerning loading the information of each specific Geo-lateral location while the developer was coding the application some latency issues were happening with regards to the API the developer was using which allows for the information to be justified and evenly spaced. Using the following class which is available to integrate with any Android project can be found on Github by bluejamesbond called TextJustify-Android. The following class although nice in styling provides text being loaded only at certain times which means that the development of this class has not been completed. With that in mind, the developer had found an alternative which is called JustifiedTextView by Navabi together while using the following class and using the Picasso API provided the needed processing power for the app not to become sluggish or not to display certain aspects of the activity.
 

\section{Efficiency of the Application}
The efficiency of the project as a whole is that is above that of what the developer expected. Considering the number of connections that could be going on at the same time regarding accessing the Firebase database and retrieving the authenticity of the users with their unique identity tags as well as processing each of the geo-locations which on opening the map activity is loaded asynchronously in such a fast and smooth manner. The efficiency of the app is produced by using worker threads in conjunction with the UI-thread of the Android device on which the application is installed this allows for multiple processes to be done simultaneously. When developing the application, the developer noticed some latency and concurrency issues while developing such an application concerning the loading of images which was done by using Picasso which allows for the downloading of images smoothly and in a synchronous type fashion, Picasso uses Automatic memory and disk caching for all images loaded with the following API. Also, async threads are also used to create a smooth feel while a user is utilising such an app. 
\par
While using the application, the device can use the mobile's data on which is fast enough for utilising an application of the following standard. Visiting different monuments within the application allows for the caching of these images and information this saves on the data usage of the application, The developer tested this by using the application over a period of time and of which the application only used 30mb of data.

\section{Usability of the Application}
A set of tasks were given to each user of the following application's quality and usability with regards to the testing of the application. The developer compiled a set of task for each user to complete. Testing of the application needed to be throughout as the developer did not want users of the application to be hit with multiple app crashes. Also, the application needed to be easily enough to use as people of all ages could use the InforMe@Dublin app. Using the program should be easily read and perform as intended as the developer promised.

\section{Tasks on using the Application for a Tester}
\begin{enumerate}
\item Logging into the InforMe@Dublin app.
\item Creation of an Account.
\item Changing notification sounds.
\item Changing power saving option.
\item Toggle focusing on devices location.
\item Adding new geofences which will be sent to the developer.
\item Using the compass of the application.
\item Choosing a geofence marker on which users can get directions using Google Maps.
\item Entering a geofence to continue onto the information of a monument.
\item Choosing text-to-speech allowing the user to listen to the information of the monument.
\item Looking at users comments.
\item Liking users comments.
\item Adding comments.
\item Removing comments.
\item Updating comments.
\item Logging out of the application. 
\end{enumerate}
\section{Tester 01}
The followings testers background is as follows. Tester one's job title is that of a graphic designer who works within the mobile phone industry within Dublin city centre. The tester is 23 years old uses mobile phones every day and has a Samsung Galaxy 6S which boasts an astonishing number of high specifications such as a whopping 3GB of Ram. The tester downloaded and installed the InforMe@Dublin application on their device and a set of tasks where given.

\subsection{Graphic Display}
It is very professional looking. The layout is clear and informative. Very impressive for a student project.

\subsection{Usability}
I found the app user-friendly. Straight forward and easy to use. Anyone of any age would be able to navigate around the app successfully.

\subsection{Additional Comments}
Would love to see more locations. Keep up the good work

\section{Tester 02}
Tester two, the next tester is a male of 48 who's occupancy is a factory worker. Working for one of Ireland's most prestigious paint companies located in Celbridge but lives in Lucan Co.Dublin. The testers device is an Alcatel Idol 4, which is a budget mobile device. The tester downloaded and installed the InforMe@Dublin application on their device and a set of tasks where given.

\subsection{Graphic Display}
The display is quite nice. Overall there is a good feel to this app.

\subsection{Usability}
It works, the app does on what the developer intended it to do. I might even use it while going on walks to learn about our area.

\subsection{Additional Comments}
Best of luck in your future endeavours.

\section{Tester 03}
Tester three's background is of a professional fitness tutor working closely with the ncef and of which is the owner of their own business within the fitness sector. The device on which this tester has used the app on is a Sony Xperia XA. The tester downloaded and installed the InforMe@Dublin application on their device and a set of tasks where given.

\subsection{Graphic Display}
The Graphical display is a superb standard compared to other apps on the play store. I find this application to be very appealing.

\subsection{Usability}
The functionality of this app is great walking around and looking at different aspects of history really makes you think and realise of what little other people had compared to what is in the same location now. The feel of the layout is quite neat.

\subsection{Additional Comments}
I would like to wish the developer of InforMe@Dublin all the best with this app. As I see it going far into the future.

\subsection{Security Testing}
The following application InforMe@Dublin uses Firebase for authentication of each user of the application. While using their email and password or Google authentication, the user of the app must create an account on the app on which they must provide their name and profile image. On creation of an account, the user is then added to the Informe Firebase database. While the user has entered the app, a check is done on which compromises of finding the unique id of the applicant in the database, if no user is found then an account is needed to be created. While testing the application regarding the login of a user concerning email and password authentication works a treat as the developer can see the database updating with the user's credentials on login and creation of an account. With regards to Google authentication, while adding the following application onto the Google play store, the developer had to retrieve the SHA-1 key used for the app release. Each API used within the development of the app a Signing-certificate fingerprint or the SHA-1 is added to each API licence.

\section{Visual Testing}
Visual testing of the InforMe@Dublin application was used to verify that the program would work as intended by the developer. While developing different parts of the application, the author would run these specific parts to see if the following code would conduce with other parts of the project to produce the outcome on which the developer was trying to achieve. While running the application, the developer could then see if the standard was paired with the code of which was written. Other standards were also needed which included exception classes to allow for errors to generate with an exception. Each part of the InforMe@Dublin project was overseen by the following method of testing on which was the most efficient, cost-effective and non-time consuming for the developer at that specific time.
\par
While also using the following testing method with the following application be developed on an Android mobile device on which allowed for the developer to use the specific developers debugging tools of which showed the usage of the CPU. Using the application as well as the GPU renderings of the application regarding the graphics of which are needed or used within the app.
\section{Summary of Chapter}
Regarding the following chapter, the developer of the InforMe@Dublin Android application has discussed the different methods and problems of which arrived while completing the application. Different test methods used such as real user experience as well as using the developers own initiative in regards to which method does what and following the methodology of which was chosen by the developer. The selected methodology allows for only moving on from one correct coding function to another when it has been extensively validated. The next chapter in this thesis is about the future work which may be undertaken by the developer.

